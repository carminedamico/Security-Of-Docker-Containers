%
% 1. process this file with pdflatex
% 2. remind to process it twice otherwise cross-references will be wrong
%
\documentclass[a4paper,12pt]{article}
%
% This is to create hyperlinks for index, URLs and citations (now we can use the
% command \url{...} to create URL with hyperlink)
%
\usepackage{color}
\usepackage[a4paper,colorlinks=true,urlcolor=blue,citecolor=blue,linkcolor=blue,bookmarks=false]{hyperref}
%
% This allows inclusion of pictures. Create figures with PowerPoint and then
% export them individually in PDF, PNG, JPEG, or GIF format (in order of
% preference)
%
\usepackage[pdftex]{graphicx}
\DeclareGraphicsExtensions{.pdf,.png,.jpg,.gif}
%
% phantom space (for abbreviations)
%
\usepackage{xspace}
%
% Definition of margins
%
\usepackage[top=2cm,bottom=2cm,left=2cm,right=2cm]{geometry}
%
% This is needed if you write the report in Italian
%
\usepackage[latin1]{inputenc}% IMPORTANTE! usare codifica ISO-8859-1 per le lettere accentate
%
% Paragraph skip and indent
%
\setlength\parskip{\medskipamount}
\setlength\parindent{0pt}
%
% Frequently used abbreviations.
% - example1: \ie this is an example
% - example2: the \ipsec protocol
%
\def\eg{e.g.\xspace}
\def\ie{i.e.\xspace}
\def\ipsec{IPsec\xspace}
\def\myfig#1{Fig.~#1\xspace}
\def\mytab#1{Tab.~#1\xspace}
\def\rfc#1{RFC-#1\xspace}% usage: \rfc{1422}
%
\begin{document}

\title{Security of Docker containers \\
{\normalsize Report for the Computer Security exam at the Politecnico di Torino}
} \author{Carmine D'Amico (239540) \\
{\normalsize tutor: Antonio Lioy} }
\date{July 2018}
\maketitle

\vfill

\rule{\textwidth}{1pt}

\tableofcontents

\rule{\textwidth}{1pt}

\vfill

\newpage

\section{Introduction}

Explain here why the XYZ protocol is important and what was the purpose of the
present work.

If you want to reference a web site you can do like this:
\url{http://www.polito.it}.

\newpage

\section{State of the art}

In computer science, the term \textit{virtualisation} is referred to the
creation of virtual computational resources. These resources, normally supplied
as hardware, are provided instead to the user by the operating system through
the creation of a new abstraction layer. OSs, storage devices or network
resources could all be virtualised. Virtualisation can be obtained at different
levels and using different techniques. \par\textit{Virtual machines} have
represented for many years the state of the art of the virtualisation, being
used in both consumer and enterprise contexts. In the last years a new
technology, based on \textit{containers}, has started to gain more attention,
thanks to its benefits. Docker is an open source container technology, stepped
into the limelight thanks to its simple interface, which allows to create and
control containers. 

\subsection{From virtual machines...}

With the term virtual machines it is often intended an \textit{hypervisor}-based
virtualisation, that is a type of virtualisation that acts at hardware level.
Virtual machines (VMs) are established on top of the host operating system,
providing applications with their dependencies, but also an entire guest OS and
a separate kernel. One or more virtual machines can be run on the same machine.
Hypervisors are distinguished in two different types, the one that works
directly on top of the host's hardware (\textbf{bare metal hypervisor}) and the
one that is on top host's OS (\textbf{hosted hypervisor}). \par Bare metal
hypervisor provides better performances, not having the overhead of the extra
layer of the host's operating system. It manages directly hardware and the
guest's operating system. On the contrary hosted hypervisor can be manged in an
easier way, running as a normal computer program on the user's operating system.
\par As said before, the hypervisor needs to run on the user's computer, which
is defined as \textit{host machine}, while each virtual machines is called
\textit{guest machine}. It is important to remember this terminology, because it
will be used also in the following, referring to containers. 

% figure of the two types of hypervisor based virtualisation


\subsection{...to containers}

\subsection{Docker}

\subsubsection{History}

\subsubsection{Implementation}

\newpage

\section{Hardening Docker}

\begin{thebibliography}{99}
%
% Citations will be numbered according to the order in which they are listed in
% this section.
%

% sample citation from a conference
\bibitem{move2008}
P.~Papadimitratos, G.~Calandriello, J.-P.~Hubaux, A.~Lioy, ``Impact of vehicular
communication security on transportation safety'', MOVE 2008: IEEE INFOCOM-2008
workshop on Mobile Networking for Vehicular Environments, Phoenix (AZ, USA),
April 13-18, 2008, pp.~1-6 

% sample citation from a journal/magazine
\bibitem{dh}
W.~Diffie, M.E.~Hellman, ``New Directions in Cryptography'', IEEE Transactions
on Information Theory, Vol.~IT-22, No.~6, November 1976, pp.~644--654

% sample citation from a RFC
\bibitem{gloss}
R.~Shirey, \rfc{4949} ``Internet Security Glossary, Version 2'', August 2007

\end{thebibliography}

\end{document}
%
% Before delivering your report, don't forget to run a spell checker, such as
% aspell (with a UK-english dictionary)
%
